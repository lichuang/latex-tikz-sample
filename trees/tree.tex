\documentclass{minimal}
\usepackage{tikz}
\usepackage{tikz-qtree}
\usepackage[UTF8, scheme=plain]{ctex}

\begin{document}
    \begin{tikzpicture}   %创建环境
            [thick,scale=0.9, every node/.style={scale=0.8}]
            %thick,scale是整张树形图的大小,可以在0~1内调整树形图的大小
            %every node/.style={scale=0.8}是每个节点文字的大小,可以修改调整节点文字的大小。

            \node {数据中心网络}
            child {node {交换机中心架构}
                child {node {树形架构}
                }
                child [missing] {}
                child {node {平坦架构}
                }
                child [missing] {}
                child {node {非结构型架构}
                }
            }    
            child [missing] {}    %child [missing] {}充当间隔符号,隔开各个子树,可以增减其个数调整子树间距。
            child [missing] {}
            child [missing] {}    
            child [missing] {}        
            child { node {服务器中心架构}
                child {node {巨型网络架构}
                }
                child [missing] {} %child [missing] {}充当间隔符号
                child {node {标准网络架构}
                }
            }    
            child [missing] {}
            child [missing] {}    
            child [missing] {}    
            child { node {增强行架构}
                child {node {无线网络架构}
                }
                child [missing] {}
                child {node {光纤网络架构}
                }
            };
        \end{tikzpicture}

\begin{tikzpicture}[grow'=right] %grow'=上下左右,可以调整树形图开口方向
	\tikzset{level distance=60pt} %设置两极树之间的距离
	\tikzset{every tree node/.style={anchor=base west}} %每一列左对齐(缺省为居中对齐)
	\Tree
	[.动力系统 
		[.动力部分 ]
		[.电力系统 
			[.发电机 ]
			[.电力网 
				[.升降压变压器 ] %注意右方括号左边有个空格
				[.输配电线路 ]
			]
			[.电力设备 ]
		]
	]
	\end{tikzpicture}
	
\end{document}